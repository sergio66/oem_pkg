% Created 2011-08-15 Mon 12:06
\documentclass[11pt]{article}
                  \usepackage{graphicx}
         \usepackage{longtable}
         \usepackage{float}
         \usepackage{wrapfig}
         \usepackage{soul}
         \usepackage{marvosym}
         \usepackage{wasysym}
         \usepackage{latexsym}
         \usepackage{amssymb}
         \usepackage{hyperref}
%         \input /Users/strow/Tex/Templates/article_setup

\providecommand{\alert}[1]{\textbf{#1}}

\title{Change Log for Rate Fitting Code}
\author{L. Strow, S. Desouza-Machado}
\date{15 August 2011}

\begin{document}

\maketitle


\section{Notes}
\label{sec-1}


\begin{itemize}
\item Src is at \~{}/Work/Rates/Fit\_pkg
\end{itemize}
\section{Changes}
\label{sec-2}
\subsection{\textit{2011-08-15 Mon} (LLS)}
\label{sec-2_1}


\begin{itemize}
\item Jacobians now normalized.  max(sum(abs(temp))) =
  max(sum(abs(water))) = max(abs(gasi)) Jacobians.  See get\_jac.m
\item get\_jac.m no longer squeezes the jacobians for the parameters being
  fit.  This is left to retrieval.m.  This way, the jacobians
  outputted by get\_jacs.m are ``full'' and can be used by adjust$_{\mathrm{rates}}$.m
  to remove the effects of some gas using a known rate.
\item Three lambda's are now used.  gas/water/temp.  The original lambda
  is now only used if you invoke regularization only on the diagonal.
\item Test directory using new covariance matrices (L1 with larger numbers
  for i=1:6) and a true apriori\_zero = all zeros.
\end{itemize}
\subsection{\textit{2011-08-08 Mon}}
\label{sec-2_2}


\begin{itemize}
\item Removed ``cluster'' stuff.
\item Many many code changes, mostly to simplify.
\begin{itemize}
\item Large speed-up by doing multiple pinv's only once and saving
\item Assume cov input matrix is the regularization matrix!
\item Used inv instead of pinv for the gain matrix
\end{itemize}
\item Both the Jacobian file and the covariance file now must have same
  ordering.  [gases columns (1-5) T$_{\mathrm{surf}}$ wv(1:97), t(1:97)]
\item Removed lamba loop, user creates this loop externally if needed
\item For now, put gain and ak matrices in driver output
\item Fixed qrenorm to reflect that T-Jacobians were scaled by 0.01, not
  0.1
\item Changed find$_{\mathrm{cov}}$$_{\mathrm{derivoperators}}$ to remove qrenorm scaling on L matrices
\item lambda input is now not scaled by 10$^{\mathrm{(lambda)}}$
\item NOTE: testing now done with a L1 matrix (unity lambda2)
\end{itemize}
\subsection{\textit{2011-08-05 08:00}}
\label{sec-2_3}


\begin{itemize}
\item the Cluster directory allows you to put the below job on the
  cluster, using run\_tara
\end{itemize}
\subsection{\textit{2011-08-04 21:00}}
\label{sec-2_4}


\begin{itemize}
\item set\_default\_struct is used to set defaults
\item these are then overwritten by calling driver =
  override\_defaults(driver,ix); where ``ix'' is the latbin you want to
  run.  So for example a loop to do all latbins would be 
\begin{verbatim}
for ix=1:36
   clear driver; 
   run_retrieval; 
end
\end{verbatim}
\end{itemize}
\subsection{\textit{2011-08-02 05:05}}
\label{sec-2_5}


\begin{itemize}
\item jacobian file now has the qrenormalization information
\end{itemize}
\subsection{\textit{2011-08-01 21:50}}
\label{sec-2_6}

\begin{itemize}
\item driver.map\_jac2cov allows you to map indices from Jacobian to Covariance
\end{itemize}

\subsection{\textit{2011-08-01 14:14}}
\label{sec-2_7}

\begin{itemize}
\item Covariance matrix format changed; no top level structure name.
  \texttt{oem\_lls.m} changed to reflect this, see \texttt{b0}.
\item Apriori format changed; no top level structure name.
  \texttt{oem\_lls.m} changed to reflect this, see \texttt{zset}, and \texttt{zstd}.
\item Changed debugging approach, using Matlab code like:
\begin{verbatim}
if driver.debug
  addpath Debug
  debug_cov
  rmpath Debug
end
\end{verbatim}
  These scripts are in the \texttt{Debug} directory.
\item Created directory Changelog to hold this file and its various output
  formats.
\end{itemize}

\begin{itemize}
\item driver.map\_jac2cov allows you to map indices from Jacobian to Covariance
\end{itemize}

\subsection{\textit{2012-08-27 to 2012-09-27}}
\label{sec-2_8}

\begin{itemize}

\item you can now specify by how much you want to adjust ALL spectral errors (ie the delta BTs) using
 driver.oem.adjust\_spectral\_errorbars (default = 1); 
this affect the Se matrix, and therefore the reported uncertainties in retrieved params

\item the column jacs/stemp are now controlled by driver.jacobian.qstnames and driver.jacobian.qstYesOrNo; the former gives
      the names while the latter tells the code whether to turn the jac on or off. For example

\begin{verbatim}
driver.jacobian.qstnames = {'CO2trop' 'CO2strat' 'O3trop' 'O3strat' 'N2O' 'CO' 'CH4' 'CFC11' 'HDO' 'stemp'};
driver.jacobian.qstYesOrNo = [1       1          1        1         1     1     1    1       1     1];
\end{verbatim}

\item the number of layers for (Q1,Q2 .. QN,T) must be the same, and sepecified by driver.jacobian.numlays (default 97). In addition,
you can have more than one radiating gas whose profile you want to rtrieve; this is contolled by driver.jacobian.numQlays
So for example if you wish to retrieve 97 layer profiles for H2O and HDO use

\begin{verbatim}
driver.jacobian.numlays     = 97;
driver.jacobian.numQlays    = 2;                         %% in addition to water, we are adding on HDO
driver.jacobian.Q2jacindex  = 1:97;
\end{verbatim}

\item Changed the $\lambda$ parameters so you either send in one constant value for all column jacs/stemp, water vapor layers and
temperature layers OR can send in individual parameters for each of the jacs that you are sending in \\

\begin{small}
\begin{verbatim}
a) driver.oem.lambda_qst can be length(1)             in which case all  "gases" get the same diag element or
   driver.oem.lambda_qst can be length(qst)           in which case each "gases" get indiv    diag element, but matrix is diagnol
   driver.oem.lambda_qst can be [length(qst) -9999]   in which case each "gases" get indiv    diag element, matrix is not diagnol
b) driver.oem.lambda_Qn can be length(1)              in which case all  "Qn" lays get the same diag element or
   driver.oem.lambda_Qn can be length(Qn)             in which case each "Qn" lays get indiv    diag element, but matrix is diagnol
   driver.oem.lambda_Qn can be [length(Qn) -9999]     in which case each "Qn" lays get indiv    diag element, matrix is not diagnol
c) driver.oem.lambda_temp can be length(1)            in which case all  "temps" get the same diag element or
   driver.oem.lambda_temp can be length(itemp)        in which case each "temps" get indiv    diag element, but matrix is diagnol
   driver.oem.lambda_temp can be [length(itemp) -9999]in which case each "temps" get indiv    diag element, matrix is not diagnol
\end{verbatim}
\end{small}
\end{itemize}

\subsection{\textit{2012-11-30 to 2012-12-03}}
\label{sec-2_9}
Updated rodgers.m so that it accepts driver.oem.spectralcov_filename for spectral covariance
Note this meant also had to do some error shecking on inv_se in same matlab source code
This is in flux; may have to change the normalization of the pectral covariance


\end{document}
